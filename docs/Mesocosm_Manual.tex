\documentclass[]{book}
\usepackage{lmodern}
\usepackage{amssymb,amsmath}
\usepackage{ifxetex,ifluatex}
\usepackage{fixltx2e} % provides \textsubscript
\ifnum 0\ifxetex 1\fi\ifluatex 1\fi=0 % if pdftex
  \usepackage[T1]{fontenc}
  \usepackage[utf8]{inputenc}
\else % if luatex or xelatex
  \ifxetex
    \usepackage{mathspec}
  \else
    \usepackage{fontspec}
  \fi
  \defaultfontfeatures{Ligatures=TeX,Scale=MatchLowercase}
\fi
% use upquote if available, for straight quotes in verbatim environments
\IfFileExists{upquote.sty}{\usepackage{upquote}}{}
% use microtype if available
\IfFileExists{microtype.sty}{%
\usepackage{microtype}
\UseMicrotypeSet[protrusion]{basicmath} % disable protrusion for tt fonts
}{}
\usepackage[margin=1in]{geometry}
\usepackage{hyperref}
\hypersetup{unicode=true,
            pdftitle={Mesocosm User Manual},
            pdfauthor={Dr.~Nyssa Silbiger  and Danielle Barnas},
            pdfborder={0 0 0},
            breaklinks=true}
\urlstyle{same}  % don't use monospace font for urls
\usepackage{natbib}
\bibliographystyle{apalike}
\usepackage{longtable,booktabs}
\usepackage{graphicx,grffile}
\makeatletter
\def\maxwidth{\ifdim\Gin@nat@width>\linewidth\linewidth\else\Gin@nat@width\fi}
\def\maxheight{\ifdim\Gin@nat@height>\textheight\textheight\else\Gin@nat@height\fi}
\makeatother
% Scale images if necessary, so that they will not overflow the page
% margins by default, and it is still possible to overwrite the defaults
% using explicit options in \includegraphics[width, height, ...]{}
\setkeys{Gin}{width=\maxwidth,height=\maxheight,keepaspectratio}
\IfFileExists{parskip.sty}{%
\usepackage{parskip}
}{% else
\setlength{\parindent}{0pt}
\setlength{\parskip}{6pt plus 2pt minus 1pt}
}
\setlength{\emergencystretch}{3em}  % prevent overfull lines
\providecommand{\tightlist}{%
  \setlength{\itemsep}{0pt}\setlength{\parskip}{0pt}}
\setcounter{secnumdepth}{5}
% Redefines (sub)paragraphs to behave more like sections
\ifx\paragraph\undefined\else
\let\oldparagraph\paragraph
\renewcommand{\paragraph}[1]{\oldparagraph{#1}\mbox{}}
\fi
\ifx\subparagraph\undefined\else
\let\oldsubparagraph\subparagraph
\renewcommand{\subparagraph}[1]{\oldsubparagraph{#1}\mbox{}}
\fi

%%% Use protect on footnotes to avoid problems with footnotes in titles
\let\rmarkdownfootnote\footnote%
\def\footnote{\protect\rmarkdownfootnote}

%%% Change title format to be more compact
\usepackage{titling}

% Create subtitle command for use in maketitle
\newcommand{\subtitle}[1]{
  \posttitle{
    \begin{center}\large#1\end{center}
    }
}

\setlength{\droptitle}{-2em}

  \title{Mesocosm User Manual}
    \pretitle{\vspace{\droptitle}\centering\huge}
  \posttitle{\par}
    \author{Dr.~Nyssa Silbiger and Danielle Barnas}
    \preauthor{\centering\large\emph}
  \postauthor{\par}
      \predate{\centering\large\emph}
  \postdate{\par}
    \date{2019-10-28}

\usepackage{booktabs}

\begin{document}
\maketitle

{
\setcounter{tocdepth}{1}
\tableofcontents
}
\chapter{Summary}\label{summary}

This manual describes the design, operation, and maintenance of the
mesocosm aquaria, located in the loading bay between Citrus Hall and
Eucalyptus Hall at California State University, Northridge, funded and
operated by Dr.~Nyssa Silbiger.

\chapter{Contacts}\label{contacts}

\begin{longtable}[]{@{}llll@{}}
\toprule
\begin{minipage}[b]{0.06\columnwidth}\raggedright\strut
Name\strut
\end{minipage} & \begin{minipage}[b]{0.06\columnwidth}\raggedright\strut
Involvement\strut
\end{minipage} & \begin{minipage}[b]{0.06\columnwidth}\raggedright\strut
Contact Info\strut
\end{minipage} & \begin{minipage}[b]{0.06\columnwidth}\raggedright\strut
Notes\strut
\end{minipage}\tabularnewline
\midrule
\endhead
\begin{minipage}[t]{0.06\columnwidth}\raggedright\strut
Dr.~Nyssa Silbiger\strut
\end{minipage} & \begin{minipage}[t]{0.06\columnwidth}\raggedright\strut
System Design Asst. Professor, CSUN\strut
\end{minipage} & \begin{minipage}[t]{0.06\columnwidth}\raggedright\strut
\href{mailto:nyssa.silbiger@csun.edu}{\nolinkurl{nyssa.silbiger@csun.edu}}
818-677-4427\strut
\end{minipage} & \begin{minipage}[t]{0.06\columnwidth}\raggedright\strut
\strut
\end{minipage}\tabularnewline
\begin{minipage}[t]{0.06\columnwidth}\raggedright\strut
Danielle Barnas\strut
\end{minipage} & \begin{minipage}[t]{0.06\columnwidth}\raggedright\strut
System Installation and Maintenance Silbiger Lab Tech, CSUN\strut
\end{minipage} & \begin{minipage}[t]{0.06\columnwidth}\raggedright\strut
\href{mailto:danielle.barnas@csun.edu}{\nolinkurl{danielle.barnas@csun.edu}}\strut
\end{minipage} & \begin{minipage}[t]{0.06\columnwidth}\raggedright\strut
\strut
\end{minipage}\tabularnewline
\begin{minipage}[t]{0.06\columnwidth}\raggedright\strut
Louis Dang\strut
\end{minipage} & \begin{minipage}[t]{0.06\columnwidth}\raggedright\strut
Systems Engineer\strut
\end{minipage} & \begin{minipage}[t]{0.06\columnwidth}\raggedright\strut
\href{mailto:louis@aqualogicinc.com}{\nolinkurl{louis@aqualogicinc.com}}\strut
\end{minipage} & \begin{minipage}[t]{0.06\columnwidth}\raggedright\strut
\href{http://www.aqualogicinc.com}{www.aqualogicinc.com}\strut
\end{minipage}\tabularnewline
\begin{minipage}[t]{0.06\columnwidth}\raggedright\strut
Bill Krohmer\strut
\end{minipage} & \begin{minipage}[t]{0.06\columnwidth}\raggedright\strut
Administrative Operations\strut
\end{minipage} & \begin{minipage}[t]{0.06\columnwidth}\raggedright\strut
\href{mailto:william.krohmer@csun.edu}{\nolinkurl{william.krohmer@csun.edu}}\strut
\end{minipage} & \begin{minipage}[t]{0.06\columnwidth}\raggedright\strut
\strut
\end{minipage}\tabularnewline
\begin{minipage}[t]{0.06\columnwidth}\raggedright\strut
Science Shop\strut
\end{minipage} & \begin{minipage}[t]{0.06\columnwidth}\raggedright\strut
CSUN College of Science and Math Machine Shop\strut
\end{minipage} & \begin{minipage}[t]{0.06\columnwidth}\raggedright\strut
818-677-3055\strut
\end{minipage} & \begin{minipage}[t]{0.06\columnwidth}\raggedright\strut
Location: EH 2014 Available M-Th 0600-1630
\href{http://www.csun.edu/science-mathematics/science-shop}{www.csun.edu/Science-Shop}\strut
\end{minipage}\tabularnewline
\begin{minipage}[t]{0.06\columnwidth}\raggedright\strut
Perry Martin\strut
\end{minipage} & \begin{minipage}[t]{0.06\columnwidth}\raggedright\strut
Supervising Plumber PPM\strut
\end{minipage} & \begin{minipage}[t]{0.06\columnwidth}\raggedright\strut
\href{mailto:perry.martin@csun.edu}{\nolinkurl{perry.martin@csun.edu}}
818-677-2222\strut
\end{minipage} & \begin{minipage}[t]{0.06\columnwidth}\raggedright\strut
\strut
\end{minipage}\tabularnewline
\begin{minipage}[t]{0.06\columnwidth}\raggedright\strut
Will Moran\strut
\end{minipage} & \begin{minipage}[t]{0.06\columnwidth}\raggedright\strut
Network Engineer CSUN IT\strut
\end{minipage} & \begin{minipage}[t]{0.06\columnwidth}\raggedright\strut
\href{mailto:will.moran@csun.edu}{\nolinkurl{will.moran@csun.edu}}
818-677-6273\strut
\end{minipage} & \begin{minipage}[t]{0.06\columnwidth}\raggedright\strut
\strut
\end{minipage}\tabularnewline
\begin{minipage}[t]{0.06\columnwidth}\raggedright\strut
Willy Martinez\strut
\end{minipage} & \begin{minipage}[t]{0.06\columnwidth}\raggedright\strut
Lead Electrician PPM Electric Shop\strut
\end{minipage} & \begin{minipage}[t]{0.06\columnwidth}\raggedright\strut
\href{mailto:willy.martinez@csun.edu}{\nolinkurl{willy.martinez@csun.edu}}
818-677-6273\strut
\end{minipage} & \begin{minipage}[t]{0.06\columnwidth}\raggedright\strut
\strut
\end{minipage}\tabularnewline
\begin{minipage}[t]{0.06\columnwidth}\raggedright\strut
Neptune Systems\strut
\end{minipage} & \begin{minipage}[t]{0.06\columnwidth}\raggedright\strut
Apex Support Team\strut
\end{minipage} & \begin{minipage}[t]{0.06\columnwidth}\raggedright\strut
\href{mailto:support@neptunesystems.com}{\nolinkurl{support@neptunesystems.com}}\strut
\end{minipage} & \begin{minipage}[t]{0.06\columnwidth}\raggedright\strut
\href{http://www.neptunesystems.com}{www.neptunesystems.com}\strut
\end{minipage}\tabularnewline
\begin{minipage}[t]{0.06\columnwidth}\raggedright\strut
Dickson Lab\strut
\end{minipage} & \begin{minipage}[t]{0.06\columnwidth}\raggedright\strut
Seawater CO2 CRMs\strut
\end{minipage} & \begin{minipage}[t]{0.06\columnwidth}\raggedright\strut
\href{mailto:co2crms@ucsd.edu}{\nolinkurl{co2crms@ucsd.edu}}
858-534-2582\strut
\end{minipage} & \begin{minipage}[t]{0.06\columnwidth}\raggedright\strut
Marine Physical Lab Scripps Institution of Oceanography University of
CA, San Diego 9500 Gilman Drive La Jolla, CA 92093-0244 USA\strut
\end{minipage}\tabularnewline
\bottomrule
\end{longtable}

\hypertarget{system-details}{\chapter{System
Details}\label{system-details}}

\textbf{Contents}\\
- \protect\hyperlink{Aquaria_System_List}{\textbf{Aquaria System}}\\
-
\protect\hyperlink{Filtration_and_Recirculation_System}{\textbf{Filtration
and Recirculation System}}\\
- \protect\hyperlink{System_Operational_Sequence}{\textbf{System
Operational Sequence}}\\
- \protect\hyperlink{Apex_Connection_Series}{\textbf{Apex Connection
Series}}\\
- \protect\hyperlink{EB832_Outlet_Connections}{\textbf{EB832 Outlet
Connections}}

 \textbf{Aquaria System Component List}

\begin{itemize}
\tightlist
\item
  Experimental Tanks (21.25'' x 12.5'' x 13.5''H) - Per tank:\\
\item
  1 Submersible powerhead pump
  (\href{https://github.com/SilbigerLab/Mesocosm_User_Manual/blob/master/Manuals/Hydor_Nano_Pump.pdf}{Hydor
  Nano Koralia 240 powerhead})\\
\item
  1 200 W Heater
  (\href{https://github.com/SilbigerLab/Mesocosm_User_Manual/blob/master/Manuals/Hydor_Heater.pdf}{Hydor
  aquarium heater})\\
\item
  1 Light
  (\href{https://github.com/SilbigerLab/Mesocosm_User_Manual/blob/master/Manuals/Apex_Halo.pdf}{Halo
  Basic M-110})\\
\item
  1 Temperature probe (Apex)\\
\item
  1 pH probe (Apex)\\
\item
  1 Solenoid valve for pH (Apex)\\
\item
  3 Flow sensors (Apex, FS-25 1/4" fitting, flow rates from 3-12 GPH
  (12-45 LPH))\\
\item
  1 Main Supply line: ``N''
\item
  1 Solenoid Supply line: ``S''
\item
  1 Drain line: ``D''
\item
  1 Gate valve (solenoid for water inflow)\\
\item
  1 VDM
  (\href{https://github.com/SilbigerLab/Mesocosm_User_Manual/blob/master/Manuals/VDM_manual.pdf}{Apex
  Variable Dimming Module}, 1 unit for 4 tanks)\\
\item
  1 FMM (\href{https://www.neptunesystems.com/getstarted/fmk/}{Apex
  Fluid Metering Module})\\
\item
  1 PM1
  (\href{https://github.com/SilbigerLab/Mesocosm_User_Manual/blob/master/Manuals/PM1_manual.pdf}{Apex
  Probe Module 1})\\
\item
  1 Base Unit
  (\href{https://github.com/SilbigerLab/Mesocosm_User_Manual/blob/master/Manuals/Apex_Comprehensive_Reference_Manual.pdf}{Apex}
  processing unit, 1 unit for 4 tanks)\\
\item
  1 EB832
  (\href{https://github.com/SilbigerLab/Mesocosm_User_Manual/blob/master/Manuals/EB832_Guide.pdf}{Apex
  8-outlet EnergyBar}, 1 unit for 2 tanks)\\
\item
  1 CO2 regulator valve
  (\href{https://github.com/SilbigerLab/Mesocosm_User_Manual/blob/master/Manuals/Tunze_CO2_Regulator.pdf}{Tunze
  pH Controller Set, pressure reducing valve 7077/3})\\
\item
  1 Industrial Grade Carbon Dioxide,
  \href{https://www.airgas.com/product/Gases/Industrial-Application-Gases/Carbon-Dioxide---Industrial/p/CD\%2050}{50
  pound cylinder}
\end{itemize}

 \textbf{Filtration and Recirculation}
\href{https://github.com/SilbigerLab/Mesocosm_User_Manual/blob/master/Manuals/Filtration_Skid_Build_Package.pdf}{\textbf{System}}
\textbf{Component List}

\begin{itemize}
\tightlist
\item
  Sump (66.25'' x 31.5'' x 21''H)\\
\item
  Chiller
  (\href{https://github.com/SilbigerLab/Mesocosm_User_Manual/blob/master/Manuals/AquaLogic_Chiller.pdf}{AquaLogic
  Multi-Temp and Titan Series})\\
\item
  Heat Pump
  (\href{https://github.com/SilbigerLab/Mesocosm_User_Manual/blob/master/Manuals/AquaLogic_Chiller.pdf}{AquaLogic
  Multi-Temp and Titan Series})\\
\item
  PM1
  (\href{https://github.com/SilbigerLab/Mesocosm_User_Manual/blob/master/Manuals/PM1_manual.pdf}{Apex
  Probe Module 1})\\
\item
  PhosBan chemical filter
  (\href{https://github.com/SilbigerLab/Mesocosm_User_Manual/blob/master/Manuals/Phosban_Reactor.pdf}{PhosBan
  Reactor 550})\\
\item
  Water pump
  (\href{https://github.com/SilbigerLab/Mesocosm_User_Manual/blob/master/Manuals/Complete_Cascade.pdf}{PerformancePro
  Cascade pump})\\
\item
  UV Sterilizer (Comet Series 95 Watt Lamp)\\
\item
  CO2 Scrubber (4 outflow tubing ports)
\item
  Airstones (4 units)\\
\item
  Carbon filter cells (3 units, CF28AC,28in, ActC)\\
\item
  Mesh filter (8 units, Matala Filter Media, interchanged 4 Blue high
  density and 4 Black low density sheets)
\end{itemize}

 \textbf{System Operational Sequence}

\begin{itemize}
\tightlist
\item
  This is a closed loop system where water from each individual tank
  will recirculate back to a main holding reservoir (sump).\\
\item
  Normal High Tide operating water level is approximately 12.5``H for a
  total water volume of 14.37 gal per tank (287.4 gal total for the
  20-tank-system).\\
\item
  Normal Low Tide operating water level is approximately 4``H for a
  total water volume of 4.60 gal per tank (92.0 gal total for the
  20-tank-system).\\
\item
  Excess water volume to sump at low tide is 9.77 gal per tank (195.4
  gal total for the 20-tank-system).\\
\item
  Normal sump operating water level is 7" water in the filter cell
  compartment, which has an approximate volume of 82.32 gal. Sump
  freeboard volume is 107.39 gal.\\
\item
  Aquaria drain line is in line with a 30 gal sump pump, which will draw
  water from the aquaria drain line and pump water to the sump.
\item
  Sump is in line with a secondary holding tank for sump overflow at Low
  Tide.
\item
  195.4 gal returning to sump in a Low Tide scenario
\item
  30 gal in sump pump
\item
  107.39 gal in sump (freeboard volume)
\item
  58.01 gal necessarily pumped to secondary holding tank
\item
  Flow rate for each tank is 2-6 GPH (See
  \href{chapters/06-tidal_manipulation.md}{Tidal Manipulation} for
  specific flow rates).\\
\item
  Chiller is plumbed inline with the
  \href{https://github.com/SilbigerLab/Mesocosm_User_Manual/blob/master/Manuals/Filtration_Skid_Build_Package.pdf}{filtration
  skid} which includes mechanical/biological filtration as well as UV
  sterilization (chemical filtration).\\
\item
  One main pump recirculates the water flow throughout the experimental
  tanks and the main holding reservoir.\\
\item
  Each tank has an immersion heater that allows tank temperatures to be
  set 15 degF (8.3 degC) above the main holding tank reservoir.\\
\item
  \textbf{Note: The tank needs to be static in order to heat up to a
  desired temp. Once the temperature has been reached then it can be set
  to flow through mode.}\\
\item
  A small sumbersible powerhead in each tank provides water circulation
  throughout the tank.\\
\item
  Each tank has (2) supply lines, each with (1) flow sensor, and (1)
  gate valve in line with (1) supply line for tidal effect. Each tank
  also has (2) drain lines with (1) flow sensor. Incoming and outgoing
  flow rates have to be manually adjusted for the tidal effect.\\
\item
  outgoing tide: incoming flow rate to be lower than outgoing flow
  rate.\\
\item
  incoming tide: incoming flow rate to be greater than outgoing flow
  rate.\\
\item
  Note: The tank will not be completely empty during low tide events to
  prevent the recirculating powerhead from running dry.\\
\item
  Each tank has (1) CO2 supply line with an airstone connected to (1)
  solenoid valve to lower pH in tanks.\\
\item
  Air compressor connected to a CO2 scrubber will bubble air into the
  sump to bring pH to ambient or near-ambient conditions in the holding
  reservoir.\\
\item
  Each tank has individual LED lighting.\\
\item
  Tanks controlled via Neptune Systems Apex Controllers. Each Apex
  controls (4) tanks.\\
\item
  Controllable parameters are pH, temperature, tidal effect, and
  lighting.
\end{itemize}

 \textbf{Apex Connection Series}

\begin{itemize}
\item
  Each EnergyBar connects to the Base Unit with an AquaBus cable via the
  AquaBus Ports for power. (2) EB832 units connect to (1) Base Unit.\\
\item
  Each CO2 Solenoid valve connects to the EnergyBar via the DC24
  Accessory Port on the side of the EB832. (2) Solenoid valves connect
  in (1) EB832.\\
\item
  \begin{enumerate}
  \def\labelenumi{(\arabic{enumi})}
  \tightlist
  \item
    PM1 connects to (1) EnergyBar with an AquaBus cable via the AquaBus
    Ports, and all PM1 modules connect in series with each other for
    power.\\
  \end{enumerate}
\item
  VDM connects to the last PM1 in series with an AquaBus cable via the
  AquaBus Ports for power.\\
\item
  Temperature probes connect to the PM1 Temp Port or the Base Unit Temp
  Port. (1) Temperature probe in each PM1, and (1) Temperature probe in
  the Base Unit.\\
\item
  pH probes connect to the PM1 pH/ORP Port or the Base Unit pH/ORP Port.
  Push the BNC female connector of the probe on to the male connector
  and turn 1/4 turn clockwise to lock the connector in place. (1) pH
  probe in each PM1, and (1) pH probe in the Base Unit.\\
\item
  \href{https://github.com/SilbigerLab/Mesocosm_User_Manual/tree/394a3f7d9fed8765e4152f9fdd11d00a2ea87a93/Manuals/HALO_Quick_Start_Guide.pdf}{Halo
  light} connects to the VDM or Base Unit via the V1/V2 or V3/V4 Port.
  (2) Light connections in the VDM and (2) Light connections in the Base
  Unit.\\
\item
  \begin{enumerate}
  \def\labelenumi{(\arabic{enumi})}
  \tightlist
  \item
    FMM connects to (1) EnergyBar (whichever EB832 is not powering the
    PM1 modules) with an AquaBus cable via the AquaBus Ports, and all
    FMM connect in series with each other for power.\\
  \end{enumerate}
\item
  \begin{enumerate}
  \def\labelenumi{(\arabic{enumi})}
  \setcounter{enumi}{2}
  \tightlist
  \item
    Flow sensors connect to each FMM via (3) of the numbered ports.
  \end{enumerate}
\end{itemize}

 \textbf{EB832 Outlet Connections}

Note: Each horizontal row on an EB832 corresponds to one tank, yielding
4 outlets per aquarium. In order:

\begin{enumerate}
\def\labelenumi{\arabic{enumi}.}
\tightlist
\item
  200W Heater
\item
  Hydor Powerhead
\item
  Water supply line ``S'' Solenoid
\item
  Halo Light
\end{enumerate}

\chapter{Inventory}\label{inventory}

\textbf{Contents}\\
- \protect\hyperlink{Experimental_Mesocosm}{\textbf{Experimental
Mesocosm}} - \protect\hyperlink{Filtration}{\textbf{Filtration}} -
\protect\hyperlink{Spare_Items}{\textbf{Spare Items}}

 \textbf{Experimental Mesocosm}

\begin{longtable}[]{@{}ll@{}}
\toprule
Item & Quantity\tabularnewline
\midrule
\endhead
Experimental Aquarium & 20\tabularnewline
Aquarium Lid & 20\tabularnewline
200W Heater & 20\tabularnewline
Suction Heater Slip & 40\tabularnewline
Halo Light & 20\tabularnewline
Halo Light Power Cords & 20\tabularnewline
Hydor Powerhead & 20\tabularnewline
Temperature Probe & 20\tabularnewline
Suction Temperature Probe Slip & 20\tabularnewline
pH Probe & 20\tabularnewline
Suction pH Probe Slip & 20\tabularnewline
pH Calibration Pack \(4,7,10\) & 10\tabularnewline
Solenoid Valve & 20\tabularnewline
Solenoid Power Supply & 20\tabularnewline
Flow Sensor & 60\tabularnewline
Gate Valve & 20\tabularnewline
Inflow Seawater Tubing & 40\tabularnewline
Inflow CO2 Tubing & 20\tabularnewline
Airstone & 20\tabularnewline
Inflow Tubing Stand & 40\tabularnewline
VDM & 5\tabularnewline
FMM & 20\tabularnewline
PM1 & 16\tabularnewline
Apex Base Unit & 5\tabularnewline
Apex EB832 & 10\tabularnewline
Tunze CO2 Regulator Valve & 1\tabularnewline
180gal Sump & 1\tabularnewline
Chiller & 1\tabularnewline
Heat Pump & 1\tabularnewline
\bottomrule
\end{longtable}

 \textbf{Filtration}

\begin{longtable}[]{@{}ll@{}}
\toprule
Item & Quantity\tabularnewline
\midrule
\endhead
PhosBan Chemical Filter with Media & 1\tabularnewline
Cascade Water Pump & 1\tabularnewline
UV Sterilizer & 1\tabularnewline
UV Light & 1\tabularnewline
CO2 Scrubber & 1\tabularnewline
Airflow Tubing & 4\tabularnewline
Airstone & 4\tabularnewline
Carbon Filter Cell & 3\tabularnewline
Matala Filter, High Density (Blue) & 4\tabularnewline
Matala Filter, Low Density (Black) & 4\tabularnewline
\bottomrule
\end{longtable}

 \textbf{Spare Items}

\begin{longtable}[]{@{}ll@{}}
\toprule
Item & Quantity\tabularnewline
\midrule
\endhead
Solenoid Valve & 4\tabularnewline
Airstone & 29\tabularnewline
Halo Light Cable & 16\tabularnewline
\bottomrule
\end{longtable}

\chapter{Start-up Guide}\label{start-up-guide}

\textbf{Contents}\\
- \protect\hyperlink{Basic_Operation}{\textbf{Basic Operation}}\\
- \protect\hyperlink{Filtration}{\textbf{Filtration}}\\
- \protect\hyperlink{System}{\textbf{System}}

 \textbf{Basic Operation}

\begin{enumerate}
\def\labelenumi{\arabic{enumi}.}
\tightlist
\item
  Operating water level in the filtration sump should be 7" in the
  filter cell compartment.
\item
  Overflow water from the tanks will feed down to the outside sump pump,
  then into to the filtration skid inside the Citrus Hall Field Room.
  The water will then be pumped through the UV sterilizer and chiller,
  then back to the tanks.
\item
  Inflow from the filtration sump to the mesocosm tanks

  \begin{enumerate}
  \def\labelenumii{\arabic{enumii}.}
  \tightlist
  \item
    Water in the sump is pulled through the three carbon filters by a
    pump and pushed up into the UV filter, where is is then directed
    through a chiller chamber.
  \item
    Water from the chiller can be directed either back into the sump (t
    valve 1 is parallel with the pvc, opening flow to the sump, and t
    valve 2 is perpendicular to the pvc, closing flow to the tanks) or
    to the mesocosm tanks (t valve 1 is angled to allow partial flow to
    the sump and tanks, and t valve 2 is parallel with the pvc, opening
    flow to the tanks). T valve 1 is used to regulate the line pressure
    going back to the mesocosm tanks. The more closed, the higher the
    pressure in the line, and the more open, the lower the pressure. The
    chiller has a safety flow switch that requires a minimum flow rate
    for the chiller to operate, so the bypass valve can be used to
    regulate the chiller flow as well as the container flow.
  \end{enumerate}
\item
  Outflow from the mesocosm tanks to the filtration sump

  \begin{enumerate}
  \def\labelenumii{\arabic{enumii}.}
  \tightlist
  \item
    Water from the tanks drains to an outdoor sump pump, which will
    automatically pump water out when a certain water level is reached.
    This water can be directed either back into the sump (t valve 3 is
    opened parallel with the pvc, allowing flow to the three dump pipes
    into the filtration skid), or if you intend to drain water in the
    event of a water change or the end of an experiment, water can be
    directed to a drain port (t valve 3 is closed and t valve 4 is
    opened parallel with the pvc).
  \end{enumerate}
\item
  Overflow from sump to secondary containment

  \begin{enumerate}
  \def\labelenumii{\arabic{enumii}.}
  \tightlist
  \item
    When mesocosm water level falls from a high tide to low tide
    sequence, more water will drain to the sump than what the sump can
    individually hold. Excess water can be redirected from the sump (S1)
    to the secondary containment (S2) by opening t valve 5 (allows
    simultaneous flow of filtered, chilled water to both S2 and the
    mesocosm tanks), and t valve 6 (allows continuous flow exchange
    between S1 and S2).
  \end{enumerate}
\item
  There are two valves located along the chiller flow line

  \begin{enumerate}
  \def\labelenumii{\arabic{enumii}.}
  \tightlist
  \item
    One controls the flow directly back to the tanks\\
  \item
    The other valve is the bypass valve which diverts the flow back to
    the sump This is used to regulate the line pressure going back to
    the container. The more closed, the higher the pressure in the line,
    and the more open, the lower the pressure. The chiller has a safety
    flow switch that requires a minimum flow rate for the chiller to
    operate, so the bypass valve can be used to regulate the chiller
    flow as well as the container flow.\\
  \end{enumerate}
\item
  Before filling the tanks make sure the drain valve located under the
  tank is closed.
\item
  Fill each rack one at a time and make sure rack and filtration skid
  flows are balanced before moving on to the next rack.
\item
  Make sure the complete system reaches equilibrium in standard
  recirculation mode before setting up the tidal cycle.
\end{enumerate}

\chapter{Tidal Manipulation}\label{tidal-manipulation}

Controling the tidal cycle of each experimental tank with the Apex. This
is achieved by manipulating the incoming and outgoing flow rates of each
individual tank with the needle valves described in the
\protect\hyperlink{system-details}{System Details}, and setting the
ON/OFF time cycle of the supply line with the solenoid. The basic
procedure is outlined below.

\begin{enumerate}
\def\labelenumi{\arabic{enumi}.}
\tightlist
\item
  Set the flow rate of the supply line N{[}\#{]}FLW (without the
  solenoid), for example 10.5 Liters/Hr, by slowly turning the black
  knob counterclockwise to increase flow or clockwise to decrease flow.

  \begin{enumerate}
  \def\labelenumii{\arabic{enumii}.}
  \tightlist
  \item
    You can view rates on the Fusion dashboard.
  \item
    Note that the Apex controller has some lag time in registering the
    flow rate after the valve has been adjusted, and the delay can be up
    to 30 seconds or more, so make small adjustments and monitor the
    change on Fusion.
  \item
    Once the rate is set you should check periodically to make sure the
    rate has not changed both on Fusion and by using a graduated
    cylinder and a timer.
  \end{enumerate}
\item
  Adjust the outgoing flow rate of the drain line D{[}\#{]}FLW higher
  than the N{[}\#{]}FLW, for example 15 Liters/Hr.

  \begin{enumerate}
  \def\labelenumii{\arabic{enumii}.}
  \tightlist
  \item
    With the above condition, the outgoing flow rate is higher than the
    incoming, so this will create a low tide effect over a 5-5.25 hr
    period.
  \end{enumerate}
\item
  To create a high tide effect, change the setting of SOL-TNK-\#
  (outlets 3 and 7 on each EB832) to ON on the Fusion dashboard, then
  manually turn on and adjust the flow rate of the supply line
  S{[}\#{]}FLW , for example 10 Liters/Hr.\\
\item
  Once the S{[}\#{]}FLW is set, change the setting of SOL-TNK-\# to AUTO
  on the Fusion dashboard.

  \begin{enumerate}
  \def\labelenumii{\arabic{enumii}.}
  \tightlist
  \item
    With the above condition, the total incoming flow rate (N+S) is
    higher thatn the outgoing (D), so this will create a high tide
    effect over a 5-5.25 hr period.
  \item
    For a constant ON/OFF cycle over a 12.5 hour period, the Advanced
    program should look like the program below:
  \end{enumerate}
\end{enumerate}

Fallback ON\\
Osc 000:00/375:00/375:00 then ON

\begin{enumerate}
\def\labelenumi{\arabic{enumi}.}
\tightlist
\item
  In the event the EnergyBar loses connection with the Apex Base,
  ``Fallback ON'' will keep the solenoid open, allowing water to
  continuously flow from S{[}\#{]}FLW.\\
\item
  The oscillate (Osc) command as written will open flow from
  S{[}\#{]}FLW for 6.25 hours, initiating the High Tide scenario, then
  close for 6.25 hours, initiating the Low Tide scenario. This will
  provide the effect of two high and two low tides of a semidiurnal
  tidal cycle over a 25 hour period.

  \begin{enumerate}
  \def\labelenumii{\arabic{enumii}.}
  \tightlist
  \item
    Using the flow rates stated above, each tidal shift will last 5.25
    hours and maintain the tide height for 1 hour.
  \end{enumerate}
\end{enumerate}

For more advanced programming features, see the
\href{https://github.com/SilbigerLab/Mesocosm_User_Manual/tree/7503b88686aef920c4a4ed473b1efe37b34dae10/Manuals/Apex_Comprehensive_Reference_Manual.pdf}{Comprehensive
Manual}. Start on Page 65 for Seasonal Features and Moon cycles.

\chapter{Controlling pH}\label{controlling-ph}

The pH is controlled with the addition of CO\textsubscript{2} gas to the
system. The gas is delivered to the tank by air stone and is controlled
through the Apex Controls with a solenoid valve connected to the EB832.

\begin{enumerate}
\def\labelenumi{\arabic{enumi}.}
\tightlist
\item
  Once the CO\textsubscript{2} regulator is connected to a tank, open
  the main tank valve.
\item
  Use the pressure adjusting screw to adjust the pressure (in bar) on
  the pressure gauge. Turning \textbf{clockwise to open}, thus
  increasing pressure, while turning \textbf{counterclockwise to close},
  thus reducing pressure.
\item
  The pressure should be set to 0.5 up to 1 bar on the gauge
  (\textasciitilde{}7.5psi)
\item
  Open the fine adjustment valve to allow gas to the tank solenoid. If
  the pressure on the gauge is too high this may prevent the
  CO\textsubscript{2} solenoid from completely closing, which will
  inject excess CO\textsubscript{2} into the system.
\item
  Programming the solenoid: pH-TNK-\#
\end{enumerate}

Control type: pH Control\\
Probe name: pH\\
Fallback: OFF\\
High Value: 8.2\\
Low Value: 7.9\\
On when: High

Refer to
\href{https://github.com/SilbigerLab/Mesocosm_User_Manual/tree/7503b88686aef920c4a4ed473b1efe37b34dae10/Manuals/Apex_Comprehensive_Reference_Manual.pdf}{Comprehensive
Manual} for set point programming.

\chapter{Apex Programming Guide}\label{apex-programming-guide}

Recommendations for programming the Apex aquarium controllers designated
for the Silbiger Lab Mesocosm, located in the loading bay between Citrus
Hall and Eucalyptus Hall at California State University, Northridge.
These recommendations are for maintaining tanks at ambient conditions.
Changes should be made according to your study aims.

The following are using the numbered system of Apex\_39106, controlling
tanks 1-4. All methods are transferrable across all 5 Apex controllers
to yield the same outcome in all 20 tanks.

\textbf{Contents}\\
- \protect\hyperlink{Probes}{\textbf{Probes}}\\
- \protect\hyperlink{Modules_Outlets_and_Ports}{\textbf{Modules,
Outlets, and Ports}}\\
- \protect\hyperlink{Outlet_Setup}{\textbf{Outlet Setup in
ApexFusion}}\\
- \protect\hyperlink{Profiles}{\textbf{Profiles}}

 \textbf{Probes}

\begin{itemize}
\tightlist
\item
  TMP-1 (Base)
\item
  PH-1 (Base)
\item
  TMP-2 (PM1\_2)
\item
  PH-2 (PM1\_2)
\item
  TMP-3 (PM1\_3)
\item
  PH-3 (PM1\_3)
\item
  TMP-4 (PM1\_4)
\item
  PH-4 (PM1\_4)
\end{itemize}

 \textbf{Modules, Outlets, and Ports}

\begin{itemize}
\tightlist
\item
  Base Unit
\item
  WHITE-TNK-1
\item
  BLUE-TNK-1
\item
  WHITE-TNK-2
\item
  BLUE-TNK-2
\item
  SndAlm\_I6
\item
  SndWrn\_I7
\item
  EmailAlm\_I5
\item
  Email2Alm\_I9
\item
  EB832\_1
\item
  HEATER-1
\item
  PWRHD-1
\item
  SOL-TNK-1
\item
  LIGHT-TNK-1
\item
  PH-TNK-1
\item
  HEATER-3
\item
  PWRHD-3
\item
  SOL-TNK-3
\item
  LIGHT-TNK-3
\item
  PH-TNK-3
\item
  EB832\_2
\item
  HEATER-2
\item
  PWRHD-2
\item
  SOL-TNK-2
\item
  LIGHT-TNK-2
\item
  PH-TNK-2
\item
  HEATER-4
\item
  PWRHD-4
\item
  SOL-TNK-4
\item
  LIGHT-TNK-4
\item
  PH-TNK-4
\item
  VDM
\item
  WHITE-TNK-3
\item
  BLUE-TNK-3
\item
  WHITE-TNK-4
\item
  BLUE-TNK-4
\item
  BluLED\_11\_5
\item
  WhtLED\_11\_6
\item
  FMM\_1
\item
  S1-FLW
\item
  N1-FLW
\item
  D1-FLW
\item
  FMM\_2
\item
  S2-FLW
\item
  N2-FLW
\item
  D2-FLW
\item
  FMM\_3
\item
  S3-FLW
\item
  N3-FLW
\item
  D3-FLW
\item
  FMM\_4
\item
  S4-FLW
\item
  N4-FLW
\item
  D4-FLW
\end{itemize}

 \textbf{Outlet Setup in ApexFusion}\\
All configurations are for Control Type: Advanced

\begin{itemize}
\tightlist
\item
  HEATER-\#
\item
  Fallback OFF\\
  Set OFF\\
  If Tmp-\# \textless{} 15.0 Then ON\\
\item
  PWRHD-\#
\item
  Fallback ON\\
  Set ON\\
\item
  Alternative program is to set Control Type: Always\\
\item
  SOL-TNK-\#
\item
  Fallback ON\\
  OSC 000:00/375:00/375:00 Then ON\\
\item
  LIGHT-TNK-\#
\item
  Fallback OFF\\
  Set OFF\\
  If Sun 0/0 Then RampUp\\
  If Moon 0/0 Then ON\\
\item
  WHITE-TNK-\#
\item
  Fallback OFF\\
  Set OFF\\
  If Sun 0/0 Then ON\\
\item
  BLUE-TNK-\#
\item
  Fallback OFF\\
  Set OFF\\
  If Moon 0/0 Then RampUp\\
\item
  WhtLED\_\#
\item
  Fallback OFF\\
  Set OFF\\
  If Sun 0/0 Then ON\\
  If Tmp-\# \textgreater{} 35.0 Then OFF\\
  Min Time 030:00 Then OFF
\item
  BluLED\_\#
\item
  Fallback OFF\\
  Set OFF\\
  If Moon 0/0 Then ON\\
  If Tmp-\# \textgreater{} 35.0 Then OFF\\
  Min Time 030:00 Then OFF
\end{itemize}

 \textbf{Profiles}

\begin{itemize}
\tightlist
\item
  RampUp:
\item
  Type: Ramp
\item
  Ramp Time: 30 min
\item
  Start Intensity: 0
\item
  End Intensity: 100
\end{itemize}

\chapter{Apex Fusion}\label{apex-fusion}

To access the Silbiger Lab Fusion account, go to
\href{https://apexfusion.com}{ApexFusion.com}, click ``Get Control'' and
enter the login information:\\
Username: SilbigerLab\\
Password: silbigerlab

\textbf{Contents}\\
- \protect\hyperlink{Dashboard}{\textbf{Dashboard Display}}\\
- \protect\hyperlink{Module_Setup}{\textbf{Module Setup}}\\
- \protect\hyperlink{Outlet_Setup}{\textbf{Outlet Setup}}\\
- \protect\hyperlink{Data_Logs}{\textbf{Downloading Data Logs}} -
\protect\hyperlink{Update}{\textbf{Update System}}

 \textbf{Dashboard Display}

\begin{itemize}
\tightlist
\item
  Left column displays the current system time, Reminders, and
  Temperature and pH trending line graphs for each tank.
\item
  Middle and right columns display the Watt, Amp, and Volt readings for
  each EB832, the state of each outlet (OFF, AUTO, or ON), and the
  flowmeter readouts for the N-valve (continuous inflow), S-valve
  (controllable inflow) and D-valve (outflow) for each tank.
\item
  You can manually control the state of the outlet by moving the slide
  bar to either OFF or ON. To let the program settings control the
  outlet state, move the slide bar to AUTO.
\item
  Adjusting the Solenoid status bar to OFF will stop flow from the
  S-valve, while adjusting the status to ON will open flow from the
  S-valve. Adjusting the status to AUTO will allow your program to
  control when the Solenoid opens and closes.
\item
  To modify the dashboard (add, remove, or reorganize items), click the
  padlock icon in the upper right-hand corner then click and drag a tile
  to move it or click the ``x'' to store it in the upper tile bank.
\end{itemize}

 \textbf{Module Setup}

\begin{enumerate}
\def\labelenumi{\arabic{enumi}.}
\tightlist
\item
  Expand the Options menu and select the Modules icon to view all
  modules.
\item
  Click any module to view a summary of its connection and software
  status, rename the module, or perform an action with that module
  (Configure, Update Software, or Delete).
\end{enumerate}

 \textbf{Outlet Setup}

\begin{enumerate}
\def\labelenumi{\arabic{enumi}.}
\tightlist
\item
  Click the Outlets icon in the upper left-hand options bar.
\item
  Outlets are arranged by the name you give them, the module they're
  associated with, the type of output, and whether or not you have
  chosen to log activity.
\item
  Select an outlet to modify its name, display symbol, and program
  settings.

  \begin{enumerate}
  \def\labelenumii{\arabic{enumii}.}
  \tightlist
  \item
    To use a program template, use the ``Control Type'' dropdown menu to
    select which item you intend to use in this outlet location, then
    fill in the required information to control the outlet state.
  \item
    To write your own program, select Advanced from the dropdown menu,
    and write your progrm in the source code box that appears.
  \end{enumerate}
\item
  Once you've completed your settings and program, click the orange
  cloud icon in the upper right to send your new settings to the Apex.
\end{enumerate}

 \textbf{Downloading Data Logs}

When connected to the same network as the Apex units, you can download
data logs for the systems following the directions below.\\
\textbf{Note} To connect to the same network, you must plug your device
into one of the live ethernet cables in the Mesocosm.

Format for your internet browser URL:\\
\url{http://:/cgi-bin/outlog.xml?sdate=yymmddhhmm\&days=n}\\
\url{http://:/cgi-bin/datalog.xml?sdate=yymmddhhmm\&days=n}\\
* Accessible logs: * Outlog -- every time the Apex changes the state of
an outlet a record is written to this log. If no outlet ever changed
state you would have zero records in this log. A new log is started
daily at midnight. Log is named ``yymmdd.odat''. * Datalog -- records
probe value snapshots (Temp, pH, ORP, etc.) based on the logging
interval you define (default = 20 minutes). You can change the interval
via the Display module under Data Log -- Log Interval. A new log is
started daily at midnight. Log is named ``yymmdd.pdat''.

Examples of what to enter into your internet browser:\\
\url{http://172.24.113.25/cgi-bin/outlog.xml?sdate=191005}\\
\url{http://172.24.113.25/cgi-bin/outlog.xml?sdate=191005\&days=7}

\begin{itemize}
\tightlist
\item
  The value after date= is the start date for when you want logged
  information, and days=n yields data n days after that start date.
\item
  Hours and Minutes are optional in the date parameter.
\end{itemize}

\begin{enumerate}
\def\labelenumi{\arabic{enumi}.}
\tightlist
\item
  Following the above format, enter the unique IP address for the Apex
  containing the logs you want to access.

  \begin{enumerate}
  \def\labelenumii{\arabic{enumii}.}
  \tightlist
  \item
    Apex\_39106: 172.24.113.25 Apex\_40216: 172.24.113.22 Apex\_39952:
    172.24.113.23 Apex\_37810: 172.24.113.21 Apex\_41239: 172.24.113.24
  \end{enumerate}
\item
  Import Log Data

  \begin{enumerate}
  \def\labelenumii{\arabic{enumii}.}
  \tightlist
  \item
    Open a new Excel file and go to the Data tab
  \item
    Select From Web in the Get External Data box
  \item
    A New Web Query dialog box will open. Type or copy the url from your
    browser into the data source Address field. Click Import.
  \item
    Your XML data should be imported into your empty spreadsheet and
    automatic filters created for each column making it easy to select
    and analyze data.
  \end{enumerate}
\end{enumerate}

 \textbf{Update System}

\begin{enumerate}
\def\labelenumi{\arabic{enumi}.}
\tightlist
\item
  From the Apex List menu

  \begin{enumerate}
  \def\labelenumii{\arabic{enumii}.}
  \tightlist
  \item
    If the orange icon next to the apex name shows an upward facing
    arrow in a circle, click that icon and select ``Update Available''
    from the dropdown menu.
  \item
    The popup window will display the current AOS version and the most
    recent AOS version available for the system. If the popup says
    ``it's recommended that you update'', then make sure the Apex is
    connected via ethernet cable and click ``Update''.
  \end{enumerate}
\item
  From the Dashboard Display

  \begin{enumerate}
  \def\labelenumii{\arabic{enumii}.}
  \tightlist
  \item
    Click the down arrow next to the apex name in the upper left corner
    and select ``Network'' from the dropdown menu.
  \item
    Under ``Apex Operating System'' you can view the installed AOS and
    Available AOS. If the installed version is out of date, there will
    be an orange bar next to ``AOS Update'' reccommending you ``Update
    AOS''. Click the bar, then make sure your Apex is connected via
    ethernet cable and click ``Update''.
  \end{enumerate}
\item
  Once the system has completed the update, go to your Modules page. If
  any modules need to be updated (under ``Status'' there will be an
  error icon rather than a check mark), click that module and change the
  Configuration Action to ``Update Software'' then click the orange
  cloud icon.
\end{enumerate}

\chapter{Breaker Box Connections}\label{breaker-box-connections}

\textbf{Following switches from top-down, then left-right:}

1,3: Main Disconnect\\
5,7: Air Conditioner\\
9,11: Condenser (Chiller, Filtration)\\
2: General Power (Lights and Outlet Box \#1)\\
4: Outlet Box \#2 (Tanks 17-20)\\
6: Outlet Box \#3 (Tanks 13-16)\\
8: Outlet Box \#4 (Tanks 9-12)\\
10: Outlet Box \#5 (Tanks 5-8)\\
12: Outlet Box \#6 (Tanks 1-4, outdoor sump pump)

\textbf{Powering the container:}

\begin{enumerate}
\def\labelenumi{\arabic{enumi}.}
\tightlist
\item
  Turn on (flip left to right) the Main Disconnect switches. Once
  powered, the remaining switches will supply power to their individual
  breakers.
\item
  Turn on (flip right to left) the General Power switch. Once powered,
  the light switch to the right of the entrance will turn on/off the
  overhead light, the O2 sensor above the light switch will be
  activated, and all outlets in Outlet Box \#1 will be active.
\item
  Turn on (flip left to right) the Air Conditioner switches. Once
  powered, the A/C unit can be controlled via remote control or the
  front display panel on the unit.
\item
  Switches 4, 6, 8, 10, and 12 all correspond to outlet boxes lining the
  upper perimeter of the container. Each box is used to power up to two
  (2) Apex units and enough modules and other devices for up to four (4)
  tanks. Each outlet covering has a number corresponding to the labeling
  for these switches. To turn any or all on, flip the switch(es) right
  to left.
\item
  Leave the Condenser switches (9 and 11) in the ``off'' position unless
  the chiller and filtration system become connected
\end{enumerate}

\chapter{Troubleshooting Guide}\label{troubleshooting-guide}

\textbf{Contents}\\
- \protect\hyperlink{Tripped_Breaker}{\textbf{Tripped Breaker}}\\
- \protect\hyperlink{Flowmeter_Misreadings}{\textbf{Flowmeter
Misreadings}}\\
- \protect\hyperlink{}{** **}\\
- \protect\hyperlink{}{** **}

 \textbf{Tripped Breaker} * Breaker tripped in the field room (pump, UV
filter, CO2 scrubber all off) 1. Call PPM (ext. 2222) and let them know
a breaker tripped at Citrus Hall in the room next to the Mechanics Room
on the outside and South side of Citrus. 1. When PPM arrives, let them
know the breaker box is on the 3rd floor of Citrus in Room 3303 and the
switch is for EDP C-4 1. Check for other items plugged into EDP C-4
outlets and alert PPM of anything that shouldn't be plugged into those
outlets (e.g.~CSUN golf carts)

 \textbf{Flowmeter Misreadings} * Flowmeter on ApexFusion is reading 0
or some incorrect value * Sometimes the flowmeters (FM) inline with
water flow will either have a bubble or some debris affecting the spin
of the turbine within the FM. 1. Make sure to check the flowmeter
connections back at the FMM module for a connection issue. Unplug and
plug back in the cable for the FM in question. 1. Try clearing bubbles.
If gently tapping the FM doesn't resolve the issue, you may have to
remove the FM to clear it out. *Removing the FM for cleaning 1. Unscrew
the FM at both of its compression fittings until the fittings are loose
on the tubing, then pull gently at the tubings to remove them from the
FMM (if the compression fitting is all the way unscrewed, but the tube
isn't coming out of the FMM, pull a little harder beacuse sometimes the
tubing just gets stuck in the FM). 1. Visually expect the inside of the
FM, and if needed, use a small long object to probe and spin the
internal turbine to dislodge any debris.

 **\_**

 **\_**

\bibliography{book.bib,packages.bib}


\end{document}
