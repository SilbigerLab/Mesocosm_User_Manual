\documentclass[]{book}
\usepackage{lmodern}
\usepackage{amssymb,amsmath}
\usepackage{ifxetex,ifluatex}
\usepackage{fixltx2e} % provides \textsubscript
\ifnum 0\ifxetex 1\fi\ifluatex 1\fi=0 % if pdftex
  \usepackage[T1]{fontenc}
  \usepackage[utf8]{inputenc}
\else % if luatex or xelatex
  \ifxetex
    \usepackage{mathspec}
  \else
    \usepackage{fontspec}
  \fi
  \defaultfontfeatures{Ligatures=TeX,Scale=MatchLowercase}
\fi
% use upquote if available, for straight quotes in verbatim environments
\IfFileExists{upquote.sty}{\usepackage{upquote}}{}
% use microtype if available
\IfFileExists{microtype.sty}{%
\usepackage{microtype}
\UseMicrotypeSet[protrusion]{basicmath} % disable protrusion for tt fonts
}{}
\usepackage[margin=1in]{geometry}
\usepackage{hyperref}
\hypersetup{unicode=true,
            pdftitle={Mesocosm User Manual},
            pdfauthor={Dr.~Nyssa Silbiger  and Danielle Barnas},
            pdfborder={0 0 0},
            breaklinks=true}
\urlstyle{same}  % don't use monospace font for urls
\usepackage{natbib}
\bibliographystyle{apalike}
\usepackage{longtable,booktabs}
\usepackage{graphicx,grffile}
\makeatletter
\def\maxwidth{\ifdim\Gin@nat@width>\linewidth\linewidth\else\Gin@nat@width\fi}
\def\maxheight{\ifdim\Gin@nat@height>\textheight\textheight\else\Gin@nat@height\fi}
\makeatother
% Scale images if necessary, so that they will not overflow the page
% margins by default, and it is still possible to overwrite the defaults
% using explicit options in \includegraphics[width, height, ...]{}
\setkeys{Gin}{width=\maxwidth,height=\maxheight,keepaspectratio}
\IfFileExists{parskip.sty}{%
\usepackage{parskip}
}{% else
\setlength{\parindent}{0pt}
\setlength{\parskip}{6pt plus 2pt minus 1pt}
}
\setlength{\emergencystretch}{3em}  % prevent overfull lines
\providecommand{\tightlist}{%
  \setlength{\itemsep}{0pt}\setlength{\parskip}{0pt}}
\setcounter{secnumdepth}{5}
% Redefines (sub)paragraphs to behave more like sections
\ifx\paragraph\undefined\else
\let\oldparagraph\paragraph
\renewcommand{\paragraph}[1]{\oldparagraph{#1}\mbox{}}
\fi
\ifx\subparagraph\undefined\else
\let\oldsubparagraph\subparagraph
\renewcommand{\subparagraph}[1]{\oldsubparagraph{#1}\mbox{}}
\fi

%%% Use protect on footnotes to avoid problems with footnotes in titles
\let\rmarkdownfootnote\footnote%
\def\footnote{\protect\rmarkdownfootnote}

%%% Change title format to be more compact
\usepackage{titling}

% Create subtitle command for use in maketitle
\newcommand{\subtitle}[1]{
  \posttitle{
    \begin{center}\large#1\end{center}
    }
}

\setlength{\droptitle}{-2em}

  \title{Mesocosm User Manual}
    \pretitle{\vspace{\droptitle}\centering\huge}
  \posttitle{\par}
    \author{Dr.~Nyssa Silbiger and Danielle Barnas}
    \preauthor{\centering\large\emph}
  \postauthor{\par}
      \predate{\centering\large\emph}
  \postdate{\par}
    \date{2019-05-03}

\usepackage{booktabs}

\begin{document}
\maketitle

{
\setcounter{tocdepth}{1}
\tableofcontents
}
\chapter{Summary}\label{summary}

This manual describes the design, operation, and maintenance of the
mesocosm aquaria, located in the loading bay between Citrus Hall and
Eucalyptus Hall at California State University, Northridge, funded and
operated by Dr.~Nyssa Silbiger.

\chapter{Contacts}\label{contacts}

\begin{longtable}[]{@{}llll@{}}
\toprule
\begin{minipage}[b]{0.06\columnwidth}\raggedright\strut
Name\strut
\end{minipage} & \begin{minipage}[b]{0.06\columnwidth}\raggedright\strut
Involvement\strut
\end{minipage} & \begin{minipage}[b]{0.06\columnwidth}\raggedright\strut
Contact Info\strut
\end{minipage} & \begin{minipage}[b]{0.06\columnwidth}\raggedright\strut
Notes\strut
\end{minipage}\tabularnewline
\midrule
\endhead
\begin{minipage}[t]{0.06\columnwidth}\raggedright\strut
Dr.~Nyssa Silbiger\strut
\end{minipage} & \begin{minipage}[t]{0.06\columnwidth}\raggedright\strut
System Design Asst. Professor, CSUN\strut
\end{minipage} & \begin{minipage}[t]{0.06\columnwidth}\raggedright\strut
\href{mailto:nyssa.silbiger@csun.edu}{\nolinkurl{nyssa.silbiger@csun.edu}}
818-677-4427\strut
\end{minipage} & \begin{minipage}[t]{0.06\columnwidth}\raggedright\strut
\strut
\end{minipage}\tabularnewline
\begin{minipage}[t]{0.06\columnwidth}\raggedright\strut
Danielle Barnas\strut
\end{minipage} & \begin{minipage}[t]{0.06\columnwidth}\raggedright\strut
System Installation and Maintenance Silbiger Lab Tech, CSUN\strut
\end{minipage} & \begin{minipage}[t]{0.06\columnwidth}\raggedright\strut
\href{mailto:danielle.barnas@csun.edu}{\nolinkurl{danielle.barnas@csun.edu}}\strut
\end{minipage} & \begin{minipage}[t]{0.06\columnwidth}\raggedright\strut
\strut
\end{minipage}\tabularnewline
\begin{minipage}[t]{0.06\columnwidth}\raggedright\strut
Louis Dang\strut
\end{minipage} & \begin{minipage}[t]{0.06\columnwidth}\raggedright\strut
Systems Engineer\strut
\end{minipage} & \begin{minipage}[t]{0.06\columnwidth}\raggedright\strut
\href{mailto:louis@aqualogicinc.com}{\nolinkurl{louis@aqualogicinc.com}}\strut
\end{minipage} & \begin{minipage}[t]{0.06\columnwidth}\raggedright\strut
\href{http://www.aqualogicinc.com}{www.aqualogicinc.com}\strut
\end{minipage}\tabularnewline
\begin{minipage}[t]{0.06\columnwidth}\raggedright\strut
Bill Krohmer\strut
\end{minipage} & \begin{minipage}[t]{0.06\columnwidth}\raggedright\strut
Administrative Operations\strut
\end{minipage} & \begin{minipage}[t]{0.06\columnwidth}\raggedright\strut
\href{mailto:william.krohmer@csun.edu}{\nolinkurl{william.krohmer@csun.edu}}\strut
\end{minipage} & \begin{minipage}[t]{0.06\columnwidth}\raggedright\strut
\strut
\end{minipage}\tabularnewline
\begin{minipage}[t]{0.06\columnwidth}\raggedright\strut
Science Shop\strut
\end{minipage} & \begin{minipage}[t]{0.06\columnwidth}\raggedright\strut
CSUN College of Science and Math Machine Shop\strut
\end{minipage} & \begin{minipage}[t]{0.06\columnwidth}\raggedright\strut
818-677-3055\strut
\end{minipage} & \begin{minipage}[t]{0.06\columnwidth}\raggedright\strut
Location: EH 2014 Available M-Th 0600-1630
\href{http://www.csun.edu/science-mathematics/science-shop}{www.csun.edu/Science-Shop}\strut
\end{minipage}\tabularnewline
\begin{minipage}[t]{0.06\columnwidth}\raggedright\strut
Perry Martin\strut
\end{minipage} & \begin{minipage}[t]{0.06\columnwidth}\raggedright\strut
Supervising Plumber PPM\strut
\end{minipage} & \begin{minipage}[t]{0.06\columnwidth}\raggedright\strut
\href{mailto:perry.martin@csun.edu}{\nolinkurl{perry.martin@csun.edu}}
818-677-2222\strut
\end{minipage} & \begin{minipage}[t]{0.06\columnwidth}\raggedright\strut
\strut
\end{minipage}\tabularnewline
\begin{minipage}[t]{0.06\columnwidth}\raggedright\strut
Will Moran\strut
\end{minipage} & \begin{minipage}[t]{0.06\columnwidth}\raggedright\strut
Network Engineer CSUN IT\strut
\end{minipage} & \begin{minipage}[t]{0.06\columnwidth}\raggedright\strut
\href{mailto:will.moran@csun.edu}{\nolinkurl{will.moran@csun.edu}}
818-677-6273\strut
\end{minipage} & \begin{minipage}[t]{0.06\columnwidth}\raggedright\strut
\strut
\end{minipage}\tabularnewline
\begin{minipage}[t]{0.06\columnwidth}\raggedright\strut
Willy Martinez\strut
\end{minipage} & \begin{minipage}[t]{0.06\columnwidth}\raggedright\strut
Lead Electrician PPM Electric Shop\strut
\end{minipage} & \begin{minipage}[t]{0.06\columnwidth}\raggedright\strut
\href{mailto:willy.martinez@csun.edu}{\nolinkurl{willy.martinez@csun.edu}}
818-677-6273\strut
\end{minipage} & \begin{minipage}[t]{0.06\columnwidth}\raggedright\strut
\strut
\end{minipage}\tabularnewline
\begin{minipage}[t]{0.06\columnwidth}\raggedright\strut
Neptune Systems\strut
\end{minipage} & \begin{minipage}[t]{0.06\columnwidth}\raggedright\strut
Apex Support Team\strut
\end{minipage} & \begin{minipage}[t]{0.06\columnwidth}\raggedright\strut
\href{mailto:support@neptunesystems.com}{\nolinkurl{support@neptunesystems.com}}\strut
\end{minipage} & \begin{minipage}[t]{0.06\columnwidth}\raggedright\strut
\href{http://www.neptunesystems.com}{www.neptunesystems.com}\strut
\end{minipage}\tabularnewline
\begin{minipage}[t]{0.06\columnwidth}\raggedright\strut
Dickson Lab\strut
\end{minipage} & \begin{minipage}[t]{0.06\columnwidth}\raggedright\strut
Seawater CO2 CRMs\strut
\end{minipage} & \begin{minipage}[t]{0.06\columnwidth}\raggedright\strut
\href{mailto:co2crms@ucsd.edu}{\nolinkurl{co2crms@ucsd.edu}}
858-534-2582\strut
\end{minipage} & \begin{minipage}[t]{0.06\columnwidth}\raggedright\strut
Marine Physical Lab Scripps Institution of Oceanography University of
CA, San Diego 9500 Gilman Drive La Jolla, CA 92093-0244 USA\strut
\end{minipage}\tabularnewline
\bottomrule
\end{longtable}

\chapter{System Details}\label{system-details}

\textbf{Contents:}\\
\href{03-system_details.md\#Tank_System}{\textbf{Aquaria System}}\\
\href{03-system_details.md\#Filtration_and_Recirculation_System}{\textbf{Filtration
and Recirculation System}}\\
\href{03-system_details.md\#System_Operational_Sequence}{\textbf{System
Operational Sequence}}\\
\href{03-system_details.md\#Apex_Connection_Series}{\textbf{Apex
Connection Series}}\\
\href{03-system_details.md\#EB832_Outlet_Connections}{\textbf{EB832
Outlet Connections}}

\textbf{Aquaria System Component List}

\begin{itemize}
\tightlist
\item
  Experimental Tanks \(21.25” x 12.5” x 13.5”H\) - Per tank:\\
\item
  1 Submersible powerhead pump
  \([Hydor Nano Koralia 240 powerhead](https://github.com/SilbigerLab/Mesocosm_User_Manual/tree/394a3f7d9fed8765e4152f9fdd11d00a2ea87a93/Manuals/Hydor_Nano_Pump.pdf)\)\\
\item
  1 200 W Heater
  \([Hydor aquarium heater](https://github.com/SilbigerLab/Mesocosm_User_Manual/tree/394a3f7d9fed8765e4152f9fdd11d00a2ea87a93/Manuals/Hydor_Heater.pdf)\)\\
\item
  1 Light
  \([Halo Basic M-110](https://github.com/SilbigerLab/Mesocosm_User_Manual/tree/394a3f7d9fed8765e4152f9fdd11d00a2ea87a93/Manuals/Apex_Halo.pdf)\)\\
\item
  1 Temperature probe \(Apex\)\\
\item
  1 pH probe \(Apex\)\\
\item
  1 Solenoid valve for pH \(Apex\)\\
\item
  3 Flow sensors
  \(Apex, FS-25 1/4" fitting, flow rates from 3-12 gph\)\\
\item
  1 Main Supply line: ``N''
\item
  1 Solenoid Supply line: ``S''
\item
  1 Drain line: ``D''
\item
  1 Gate valve \(solenoid for water inflow\)\\
\item
  1 VDM
  \([Apex Variable Dimming Module](https://github.com/SilbigerLab/Mesocosm_User_Manual/tree/394a3f7d9fed8765e4152f9fdd11d00a2ea87a93/Manuals/VDM_manual.pdf), 1 unit for 4 tanks\)\\
\item
  1 FMM
  \([Apex Fluid Metering Module](https://www.neptunesystems.com/getstarted/fmk/)\)\\
\item
  1 PM1
  \([Apex Probe Module 1](https://github.com/SilbigerLab/Mesocosm_User_Manual/tree/394a3f7d9fed8765e4152f9fdd11d00a2ea87a93/Manuals/PM1_manual.pdf)\)\\
\item
  1 Base Unit
  \([Apex](https://github.com/SilbigerLab/Mesocosm_User_Manual/tree/394a3f7d9fed8765e4152f9fdd11d00a2ea87a93/Manuals/Apex_Comprehensive_Reference_Manual.pdf) processing unit, 1 unit for 4 tanks\)\\
\item
  1 EB832
  \([Apex 8-outlet EnergyBar](https://github.com/SilbigerLab/Mesocosm_User_Manual/tree/394a3f7d9fed8765e4152f9fdd11d00a2ea87a93/Manuals/EB832_Guide.pdf), 1 unit for 2 tanks\)\\
\item
  1 CO2 regulator valve
  \([Tunze pH Controller Set, pressure reducing valve 7077/3](https://github.com/SilbigerLab/Mesocosm_User_Manual/tree/394a3f7d9fed8765e4152f9fdd11d00a2ea87a93/Manuals/Tunze_CO2_Regulator.pdf)\)\\
\item
  1 Industrial Grade Carbon Dioxide,
  \href{https://www.airgas.com/product/Gases/Industrial-Application-Gases/Carbon-Dioxide---Industrial/p/CD\%2050}{50
  pound cylinder}
\end{itemize}

\textbf{Filtration and Recirculation}
\href{https://github.com/SilbigerLab/Mesocosm_User_Manual/tree/394a3f7d9fed8765e4152f9fdd11d00a2ea87a93/Manuals/Filtration_Skid_Build_Package.pdf}{\textbf{System}}
\textbf{Component List}

\begin{itemize}
\tightlist
\item
  Sump \(66.25” x 31.5” x 21”H\)\\
\item
  Chiller
  \([AquaLogic Multi-Temp and Titan Series](https://github.com/SilbigerLab/Mesocosm_User_Manual/tree/394a3f7d9fed8765e4152f9fdd11d00a2ea87a93/Manuals/AquaLogic_Chiller.pdf)\)\\
\item
  Heat Pump
  \([AquaLogic Multi-Temp and Titan Series](https://github.com/SilbigerLab/Mesocosm_User_Manual/tree/394a3f7d9fed8765e4152f9fdd11d00a2ea87a93/Manuals/AquaLogic_Chiller.pdf)\)\\
\item
  PM1 \(Apex Probe Module 1\)\\
\item
  PhosBan chemical filter
  \([PhosBan Reactor 550](https://github.com/SilbigerLab/Mesocosm_User_Manual/tree/394a3f7d9fed8765e4152f9fdd11d00a2ea87a93/Manuals/Phosban_Reactor.pdf)\)\\
\item
  Water pump
  \([PerformancePro Cascade pump](https://github.com/SilbigerLab/Mesocosm_User_Manual/tree/394a3f7d9fed8765e4152f9fdd11d00a2ea87a93/Manuals/Complete_Cascade.pdf)\)\\
\item
  UV Sterilizer \(Comet Series 95 Watt Lamp\)\\
\item
  CO2 Scrubber \(4 outflow tubing ports\)
\item
  Airstones \(4 units\)\\
\item
  Carbon filter cells \(3 units, CF28AC,28in, ActC\)\\
\item
  Mesh filter
  \(8 units, Matala Filter Media, interchanged 4 Blue high density and 4 Black low density sheets\)
\end{itemize}

\textbf{System Operational Sequence}

\begin{itemize}
\tightlist
\item
  This is a closed loop system where water from each individual tank
  will recirculate back to a main holding reservoir \(sump\).\\
\item
  Normal High Tide operating water level is approximately 12.5``H for a
  total water volume of 14.37 gal per tank
  \(287.4 gal total for the 20-tank-system\).\\
\item
  Normal Low Tide operating water level is approximately 4``H for a
  total water volume of 4.60 gal per tank
  \(92.0 gal total for the 20-tank-system\).\\
\item
  Excess water volume to sump at low tide is 9.77 gal per tank
  \(195.4 gal total for the 20-tank-system\).\\
\item
  Normal sump operating water level is 7" water in the filter cell
  compartment, which has an approximate volume of 82.32 gal. Sump
  freeboard volume is 107.39 gal.\\
\item
  Aquaria drain line is in line with a 30 gal sump pump, which will draw
  water from the aquaria drain line and pump water to the sump.
\item
  Sump is in line with a secondary holding tank for sump overflow at Low
  Tide.
\item
  195.4 gal returning to sump in a Low Tide scenario
\item
  30 gal in sump pump
\item
  107.39 gal in sump \(freeboard volume\)
\item
  58.01 gal necessarily pumped to secondary holding tank
\item
  Flow rate for each tank is 2-6 GPH.\\
\item
  Chiller is plumbed inline with the
  \href{https://github.com/SilbigerLab/Mesocosm_User_Manual/tree/394a3f7d9fed8765e4152f9fdd11d00a2ea87a93/Manuals/Filtration_Skid_Build_Package.pdf}{filtration
  skid} which includes mechanical/biological filtration as well as UV
  sterilization \(chemical filtration\).\\
\item
  One main pump recirculates the water flow throughout the experimental
  tanks and the main holding reservoir.\\
\item
  Each tank has an immersion heater that allows tank temperatures to be
  set 15 degF \(8.3 degC\) above the main holding tank reservoir.\\
\item
  \textbf{Note: The tank needs to be static in order to heat up to a
  desired temp. Once the temperature has been reached then it can be set
  to flow through mode.}\\
\item
  A small sumbersible powerhead in each tank provides water circulation
  throughout the tank.\\
\item
  Each tank has \(2\) supply lines, each with \(1\) flow sensor, and
  \(1\) gate valve in line with \(1\) supply line for tidal effect. Each
  tank also has \(2\) drain lines with \(1\) flow sensor. Incoming and
  outgoing flow rates have to be manually adjusted for the tidal
  effect.\\
\item
  outgoing tide: incoming flow rate to be lower than outgoing flow
  rate.\\
\item
  incoming tide: incoming flow rate to be greater than outgoing flow
  rate.\\
\item
  Note: The tank will not be completely empty during low tide events to
  prevent the recirculating powerhead from running dry.\\
\item
  Each tank has \(1\) CO2 supply line with an airstone connected to
  \(1\) solenoid valve to lower pH in tanks.\\
\item
  Air compressor connected to a CO2 scrubber will bubble air into the
  sump to bring pH to ambient or near-ambient conditions in the holding
  reservoir.\\
\item
  Each tank has individual LED lighting.\\
\item
  Tanks controlled via Neptune Systems Apex Controllers. Each Apex
  controls \(4\) tanks.\\
\item
  Controllable parameters are pH, temperature, tidal effect, and
  lighting.
\end{itemize}

\textbf{Apex Connection Series}

\begin{itemize}
\tightlist
\item
  Each EnergyBar connects to the Base Unit with an AquaBus cable via the
  AquaBus Ports for power. \(2\) EB832 units connect to \(1\) Base
  Unit.\\
\item
  Each CO2 Solenoid valve connects to the EnergyBar via the DC24
  Accessory Port on the side of the EB832. \(2\) Solenoid valves connect
  in \(1\) EB832.\\
\item
  \(1\) PM1 connects to \(1\) EnergyBar with an AquaBus cable via the
  AquaBus Ports, and all PM1 modules connect in series with each other
  for power.\\
\item
  VDM connects to the last PM1 in series with an AquaBus cable via the
  AquaBus Ports for power.\\
\item
  Temperature probes connect to the PM1 Temp Port or the Base Unit Temp
  Port. \(1\) Temperature probe in each PM1, and \(1\) Temperature probe
  in the Base Unit.\\
\item
  pH probes connect to the PM1 pH/ORP Port or the Base Unit pH/ORP Port.
  Push the BNC female connector of the probe on to the male connector
  and turn 1/4 turn clockwise to lock the connector in place. \(1\) pH
  probe in each PM1, and \(1\) pH probe in the Base Unit.\\
\item
  \href{https://github.com/SilbigerLab/Mesocosm_User_Manual/tree/394a3f7d9fed8765e4152f9fdd11d00a2ea87a93/Manuals/HALO_Quick_Start_Guide.pdf}{Halo
  light} connects to the VDM or Base Unit via the V1/V2 or V3/V4 Port.
  \(2\) Light connections in the VDM and \(2\) Light connections in the
  Base Unit.\\
\item
  \(1\) FMM connects to \(1\) EnergyBar
  \(whichever EB832 is not powering the PM1 modules\) with an AquaBus
  cable via the AquaBus Ports, and all FMM connect in series with each
  other for power.\\
\item
  \(3\) Flow sensors connect to each FMM via \(3\) of the numbered
  ports.
\end{itemize}

\textbf{EB832 Outlet Connections}

Note: Each horizontal row on an EB832 corresponds to one tank, yielding
4 outlets per aquarium. In order:

\begin{enumerate}
\def\labelenumi{\arabic{enumi}.}
\tightlist
\item
  200W Heater
\item
  Hydor Powerhead
\item
  Water supply line ``S'' Solenoid
\item
  Halo Light
\end{enumerate}

\chapter{Inventory}\label{inventory}

\textbf{Experimental Mesocosm}

\begin{longtable}[]{@{}ll@{}}
\toprule
Item & Quantity\tabularnewline
\midrule
\endhead
Experimental Aquarium & 20\tabularnewline
Aquarium Lid & 20\tabularnewline
200W Heater & 20\tabularnewline
Suction Heater Slip & 40\tabularnewline
Halo Light & 20\tabularnewline
Halo Light Power Cords & 20\tabularnewline
Hydor Powerhead & 20\tabularnewline
Temperature Probe & 20\tabularnewline
Suction Temperature Probe Slip & 20\tabularnewline
pH Probe & 20\tabularnewline
Suction pH Probe Slip & 20\tabularnewline
pH Calibration Pack \(4,7,10\) & 10\tabularnewline
Solenoid Valve & 20\tabularnewline
Solenoid Power Supply & 20\tabularnewline
Flow Sensor & 60\tabularnewline
Gate Valve & 20\tabularnewline
Inflow Seawater Tubing & 40\tabularnewline
Inflow CO2 Tubing & 20\tabularnewline
Airstone & 20\tabularnewline
Inflow Tubing Stand & 40\tabularnewline
VDM & 5\tabularnewline
FMM & 20\tabularnewline
PM1 & 16\tabularnewline
Apex Base Unit & 5\tabularnewline
Apex EB832 & 10\tabularnewline
Tunze CO2 Regulator Valve & 1\tabularnewline
180gal Sump & 1\tabularnewline
Chiller & 1\tabularnewline
Heat Pump & 1\tabularnewline
\bottomrule
\end{longtable}

\textbf{Filtration}

\begin{longtable}[]{@{}ll@{}}
\toprule
Item & Quantity\tabularnewline
\midrule
\endhead
PhosBan Chemical Filter with Media & 1\tabularnewline
Cascade Water Pump & 1\tabularnewline
UV Sterilizer & 1\tabularnewline
UV Light & 1\tabularnewline
CO2 Scrubber & 1\tabularnewline
Airflow Tubing & 4\tabularnewline
Airstone & 4\tabularnewline
Carbon Filter Cell & 3\tabularnewline
Matala Filter, High Density \(Blue\) & 4\tabularnewline
Matala Filter, Low Density \(Black\) & 4\tabularnewline
\bottomrule
\end{longtable}

\textbf{Spare Items}

\begin{longtable}[]{@{}ll@{}}
\toprule
Item & Quantity\tabularnewline
\midrule
\endhead
Solenoid Valve & 4\tabularnewline
Airstone & 29\tabularnewline
Halo Light Cable & 16\tabularnewline
\bottomrule
\end{longtable}

\chapter{Start-up Guide}\label{start-up-guide}

\begin{enumerate}
\def\labelenumi{\arabic{enumi}.}
\tightlist
\item
  Operating water level in the filtration sump should be 7" in the
  filter cell compartment.
\item
  Water from the container will feed back to the filtration skid, and
  from there it is pumped through the UV sterilizer and chiller barrel,
  then back to the container.
\item
  There are two valves located after the chiller
\end{enumerate}

\begin{enumerate}
\def\labelenumi{\alph{enumi}.}
\item
  One controls the flow directly back to the tanks
\item
  The other valve is the bypass valve which diverts the flow back to the
  sump This is used to regulate the line pressure going back to the
  container. The more closed, the higher the pressure in the line, and
  the more open, the lower the pressure. The chiller has a safety flow
  switch that requires a minimum flow rate for the chiller to operate,
  so the bypass valve is used in this case to regulate the chiller flow
  as well as the container flow.
\end{enumerate}

\begin{enumerate}
\def\labelenumi{\arabic{enumi}.}
\setcounter{enumi}{3}
\tightlist
\item
  Before filling the tanks make sure the drain valve located under the
  tank is closed.
\item
  Fill each rack one at a time and make sure rack and filtration skid
  flows are balanced before moving on to the next rack.
\item
  Make sure the complete system reaches equilibrium in standard
  recirculation mode before setting up the tidal cycle.
\end{enumerate}

\chapter{Tidal Manipulation}\label{tidal-manipulation}

Controling the tidal cycle of each experimental tank with the Apex. This
is achieved by manipulating the incoming and outgoing flow rates of each
individual tank with the needles described in the \[System Details\],
and setting the ON/OFF time cycle of the supply line with the solenoid.
The basic procedure is outlined below.

\begin{enumerate}
\def\labelenumi{\arabic{enumi}.}
\tightlist
\item
  Set the flow rate of the supply line N\[\#\]FLW, the one without the
  solenoid, for example 12 Liters/Hr, by slowly turning the black knob.
\end{enumerate}

\begin{enumerate}
\def\labelenumi{\alph{enumi}.}
\tightlist
\item
  Note that the Apex controller has some lag time in registering the
  flow rate after the valve has been adjusted, the delay can be up to 30
  seconds or more. Once the rate is set you should check periodically to
  make sure the rate has not changed using a graduated cylinder and a
  timer.
\end{enumerate}

\begin{enumerate}
\def\labelenumi{\arabic{enumi}.}
\setcounter{enumi}{1}
\tightlist
\item
  Adjust the outgoing flow rate of the drain line D\[\#\]FLW higher than
  the N\[\#\]FLW, for example 19.04 Liters/Hr.
\end{enumerate}

\begin{enumerate}
\def\labelenumi{\alph{enumi}.}
\tightlist
\item
  With the above condition, the outgoing flow rate is higher than the
  incoming, so this will create the low tide effect.
\end{enumerate}

\begin{enumerate}
\def\labelenumi{\arabic{enumi}.}
\setcounter{enumi}{2}
\tightlist
\item
  To set the high tide effect, manually turn on and adjust the flow rate
  of the supply line S\[\#\]FLW , for example 14.08 Liters/Hr.\\
\item
  Once the S\[\#\]FLW is set, change setting of SOL-TNK-\#
  \(outlets 3 and 7 on each EB832\) to AUTO on the Fusion page or using
  the Display Module. For a constant ON/OFF cycle over a 12.5 hour
  period, the Advanced program should look something like the program
  below.
\end{enumerate}

Fallback ON\\
Osc 000:00/375:00/375:00 then ON

\begin{enumerate}
\def\labelenumi{\arabic{enumi}.}
\tightlist
\item
  In the event the EnergyBar loses connection with the Apex Base,
  Fallback: ON will keep the solenoid open, allowing water to
  continuously flow from S\[\#\]FLW.\\
\item
  The oscillate command as written will open flow from S\[\#\]FLW for
  6.25 hours, initiating the High Tide scenario, then close for 6.25
  hours, initiating the Low Tide scenario. This will provide the effect
  of two high and two low tides of a semidiurnal tidal cycle over a 25
  hour period.\\
\item
  Using the flow rates stated above, each tidal shift will last 5.25
  hours and maintain the tide for 1 hour.
\end{enumerate}

For more advanced programming features, see the
\href{https://github.com/SilbigerLab/Mesocosm_User_Manual/tree/7503b88686aef920c4a4ed473b1efe37b34dae10/Manuals/Apex_Comprehensive_Reference_Manual.pdf}{Comprehensive
Manual}. Start on Page 65 for Seasonal Features and Moon cycles.

\chapter{Controlling pH}\label{controlling-ph}

The pH is controlled with the addition of CO2 gas to the system. The gas
is delivered to the tank by air stone and is controlled through the Apex
Controls with a solenoid valve connected to the EB832.

\begin{enumerate}
\def\labelenumi{\arabic{enumi}.}
\tightlist
\item
  Once the CO2 regulator is connected to a tank, open the main tank
  valve.
\item
  Use the pressure adjusting screw to adjust the pressure \(in bar\) on
  the pressure gauge. Turning \textbf{clockwise to open}, thus
  increasing pressure, while turning \textbf{counterclockwise to close},
  thus reducing pressure.
\item
  The pressure should be set to 0.5 up to 1 bar on the gauge \(~7.5psi\)
\item
  Open the fine adjustment valve to allow gas to the tank solenoid. If
  the pressure on the gauge is too high this may prevent the CO2
  solenoid from completely closing, which will inject excess CO2 into
  the system.
\item
  Programming the solenoid: pH-TNK-\#
\end{enumerate}

Control type: pH Control\\
Probe name: pH\\
Fallback: OFF High Value: 8.2\\
Low Value: 7.9\\
On when: High

Refer to
\href{https://github.com/SilbigerLab/Mesocosm_User_Manual/tree/7503b88686aef920c4a4ed473b1efe37b34dae10/Manuals/Apex_Comprehensive_Reference_Manual.pdf}{Comprehensive
Manual} for set point programming.

\chapter{Apex Programming Guide}\label{apex-programming-guide}

Recommendations for programming the Apex aquarium controllers designated
for the Silbiger Lab Mesocosm, located in the loading bay between Citrus
Hall and Eucalyptus Hall at California State University, Northridge.
These recommendations are for maintaining tanks at ambient conditions.
Changes should be made according to your study aims.

The following are using the numbered system of Apex\_39106, controlling
tanks 1-4. All methods are transferrable across all 5 Apex controllers
to yield the same outcome in all 20 tanks.

\textbf{Contents}\\
\href{08-apex_programming_guide.md\#Probes}{\textbf{Probes}}\\
\href{08-apex_programming_guide.md\#Outlets_and_Ports}{\textbf{Outlets
and Ports}}\\
\href{08-apex_programming_guide.md\#Outlet_Setup}{\textbf{Outlet Setup
in ApexFusion}}\\
\href{08-apex_programming_guide.md\#Profiles}{\textbf{Profiles}}

\textbf{Probes}

\begin{itemize}
\tightlist
\item
  Tmp-1 \(Base\)
\item
  pH-1 \(Base\)
\item
  Tmp-2 \(PM1\_2\)
\item
  pH-2 \(PM1\_2\)
\item
  Tmp-3 \(PM1\_3\)
\item
  pH-3 \(PM1\_3\)
\item
  Tmp-4 \(PM1\_4\)
\item
  pH-4 \(PM1\_4\)
\end{itemize}

\textbf{Outlets and Ports}

\begin{itemize}
\tightlist
\item
  Base Unit
\item
  WHITE-TNK-1
\item
  BLUE-TNK-1
\item
  WHITE-TNK-2
\item
  BLUE-TNK-2
\item
  SndAlm\_I6
\item
  SndWrn\_I7
\item
  EmailAlm\_I5
\item
  Email2Alm\_I9
\item
  EB832\_1
\item
  HEATER-1
\item
  PWRHD-1
\item
  SOL-TNK-1
\item
  LIGHT-TNK-1
\item
  PH-TNK-1
\item
  HEATER-3
\item
  PWRHD-3
\item
  SOL-TNK-3
\item
  LIGHT-TNK-3
\item
  PH-TNK-3
\item
  EB832\_2
\item
  HEATER-2
\item
  PWRHD-2
\item
  SOL-TNK-2
\item
  LIGHT-TNK-2
\item
  PH-TNK-2
\item
  HEATER-4
\item
  PWRHD-4
\item
  SOL-TNK-4
\item
  LIGHT-TNK-4
\item
  PH-TNK-4
\item
  VDM
\item
  WHITE-TNK-3
\item
  BLUE-TNK-3
\item
  WHITE-TNK-4
\item
  BLUE-TNK-4
\item
  BluLED\_11\_5
\item
  WhtLED\_11\_6
\item
  FMM\_1
\item
  S1-FLW
\item
  N1-FLW
\item
  D1-FLW
\item
  FMM\_2
\item
  S2-FLW
\item
  N2-FLW
\item
  D2-FLW
\item
  FMM\_3
\item
  S3-FLW
\item
  N3-FLW
\item
  D3-FLW
\item
  FMM\_4
\item
  S4-FLW
\item
  N4-FLW
\item
  D4-FLW
\end{itemize}

\textbf{Outlet Setup in ApexFusion}

\begin{itemize}
\tightlist
\item
  LIGHT-TNK-\#
\item
  Fallback OFF
\item
  Set OFF
\item
  If Sun 0/0 Then ON
\item
  If Moon 0/0 Then ON
\item
  WHITE-TNK-\#
\item
  Fallback OFF
\item
  Set OFF
\item
  If Sun 0/0 Then RampUp
\item
  BLUE-TNK-\#
\item
  Fallback OFF
\item
  Set OFF
\item
  If Moon 0/0 Then RampUp
\item
  WhtLED\_\#
\item
  Fallback OFF
\item
  Set OFF
\item
  If Sun 0/0 Then RampUp
\item
  BluLED\_\#
\item
  Fallback OFF
\item
  Set OFF
\item
  If Moon 0/0 Then RampUp
\end{itemize}

\textbf{Profiles}

\begin{itemize}
\tightlist
\item
  RampUp:
\item
  Ramp Time: 30 min
\item
  Start Intensity: 0
\item
  End Intensity: 100
\end{itemize}

\chapter{Apex Fusion}\label{apex-fusion}

To access the Silbiger Lab Fusion account, click
\href{https://github.com/SilbigerLab/Mesocosm_User_Manual/tree/7503b88686aef920c4a4ed473b1efe37b34dae10/Chapters/apexfusion.com}{here},
click ``Get Control'' and enter the login information:\\
Username: SilbigerLab\\
Password: silbigerlab

\textbf{Contents}\\
\href{09-apex_fusion_guide.md\#Dashboard}{\textbf{Dashboard}}\\
\href{09-apex_fusion_guide.md\#Outlet_Setup}{\textbf{Outlet Setup}}\\
\href{09-apex_fusion_guide.md\#Data_Logs}{\textbf{Downloading Data
Logs}}

\textbf{Dashboard}

\begin{itemize}
\item
  On your Dashboard you will see every outlet available to you and the
  current state of that outlet \(ON or OFF\), as well as the current
  readings for any probes enabled on the Apex.
\item
  On the left are graphical logs of Salinity, pH and Temperature in the
  first of all four tanks controlled by the Apex
\item
  The top menu bar options \(left to right\):
\item
  Apex List: returns you to the menu of available Apexes linked to your
  Fusion account
\item
  Alarm Log: shows any daily alarms triggered by the system, as set by
  your parameters
\item
  Input Log:
\item
\end{itemize}

\textbf{Outlet Setup}

\begin{enumerate}
\def\labelenumi{\arabic{enumi}.}
\tightlist
\item
  Click Expand along the top bar \(depicted as three gears\).
\item
  Immediately next to the gear symbol is the Outputs icon
  \(depicted as a three pronged outlet\). Click this icon to view your
  array of outlets and other outputs from the system, as well as
  manipulate these items. These are arranged by the name you give them,
  the device they're connected to, the type of output, and whether or
  not you have chosen to log this item's history.
\item
  Click any ``outlet'' type to
\end{enumerate}

\textbf{Downloading Data Logs}

Format of what to enter into your internet browser:\\
\url{http://:/cgi-bin/outlog.xml?sdate=yymmddhhmm\&days=n}

Examples of what to enter into your internet browser:\\
\url{http://130.166.116.174/cgi-bin/outlog.xml?sdate=190426}\\
\url{http://130.166.116.174/cgi-bin/datalog.xml?sdate=190426\&days=7}

\begin{itemize}
\tightlist
\item
  the value after sdate= is the start date for when you want logged
  information, and days=n yields data n days after that start date.
\end{itemize}

\chapter{Breaker Box Connections}\label{breaker-box-connections}

Following switches from top-down, then left-right.

1,3: Main Disconnect\\
5,7: Air Conditioner\\
9,11: Condenser (Chiller, Filtration)\\
2: General Power (Lights and Outlet Box \#1)\\
4: Outlet Box \#2 (Tanks 17-20)\\
6: Outlet Box \#3 (Tanks 13-16)\\
8: Outlet Box \#4 (Tanks 9-12)\\
10: Outlet Box \#5 (Tanks 5-8)\\
12: Outlet Box \#6 (Tanks 1-4)

Powering the container: 1. Turn on (flip left to right) the Main
Disconnect switches. Once powered, the remaining switches will supply
power to their individual breakers. 1. Turn on (flip right to left) the
General Power switch. Once powered, the light switch to the right of the
entrance will turn on/off the overhead light, the O2 sensor above the
light switch will be activated, and all outlets in Outlet Box \#1 will
be active. 1. Turn on (flip left to right) the Air Conditioner switches.
Once powered, the A/C unit can be controlled via remote control or the
front display panel on the unit. 1. Switches 4, 6, 8, 10, and 12 all
correspond to outlet boxes lining the upper perimeter of the container.
Each box is used to power up to two (2) Apex units and enough modules
and other devices for up to four (4) tanks. Each outlet covering has a
number corresponding to the labeling for these switches. To turn any or
all on, flip the switch(es) right to left. 1. Leave the Condenser
switches (9 and 11) in the ``off'' position unless the chiller and
filtration system become connected

\bibliography{book.bib,packages.bib}


\end{document}
